
\subsection{The bootstrap method}
In order to find the errors of the mean value of our Monte-Carlo integration the bootstrap method is used. It has the advantage that it requires no information about the distribution of our results. Instead the \textit{empirical distribution} is used which is determined from the samples themselves. \\
To perform the bootstrap we take our sample of results ($x_1$,...,$x_n$) and randomly create $m$ so called bootstrap samples ($x^{\star}_1$,...,$x^{\star}_n$) from it by permuting the $x_i$ of our original sample with replacement. Then, for each bootstrap sample $x^{\star}$ we take an estimator of the mean value. This mean value is called a \textit{bootstrap replica} of our mean value. From these $m$ bootstrap replica we can calculate the standard deviation over all of them
\begin{equation}
	sd(M)=\sqrt{\frac{\sum_{i=1}^{m}(M^{\star}_i-M^{\star})^2}{m-1}} \label{bootstraperror}
\end{equation}  
Here $M$ is the mean value for which the error was desired and $M^{\star}=\frac{1}{m}\sum_{i=1}^{m}M^{\star}_i$ is the usual mean value of the bootstrap samples \cite{computational}.
	
\subsection{Metropolis-Hastings (failed attempt)}
The following method does not work for the Hubbard-Model, but since we spent a lot of time on it, we want to introduce it anyways.
The Metropolis-Hasting algorithm is a variant of Markov Chain Monte-Carlo where first an initial state of the system is randomly chosen.
For example the electron configuration of a lattice. Then the system gets repeatedly the chance to change its state by spin flips, electron hops, etc. This change will be, similarly to simulated annealing, accepted with a probability \eqref{acceptance} that depends on the energy difference that the change of state causes and the inverse temperature $\beta$.
\begin{equation}\label{acceptance}
p_{acceptance} = \min(1,\exp(-\beta*\Delta H))
\end{equation}
The resulting set of states is a weighted representation of the full system as it prefers physically favorable states.
Therefore the expectation value of an operator can be approximated by averaging over the result for each chosen state.
This method works well for the Ising-Model, however it can not be applied to the Hubbard-Model, which has no diagonal Hamiltonian and therefore non trivial eigenstates, which are superpositions of an exponentially large number of basisstates.
The energy(difference), which is crucial for the Metropolis algorithm, is hence not well defined for the different electron configurations.