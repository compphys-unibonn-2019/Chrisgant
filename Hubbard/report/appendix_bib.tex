\documentclass[nenglish]{scrartcl} 
\KOMAoptions{fontsize=12pt,paper=a4}       
\KOMAoptions{DIV=11}                      
\usepackage[utf8]{inputenc}                
\usepackage[T1]{fontenc}                   
\usepackage{babel}   
\usepackage{amsmath} 
\setcounter{MaxMatrixCols}{20}                    
\usepackage[autostyle=true]{csquotes}     

\begin{thebibliography}{9}
	
	\bibitem{luu}
	Thomas Luu, November 2017
	\textit{Fermions and Computers}
	
	\bibitem{lieb}
	E. Lieb, F. Wu
	\textit{The one-dimensional Hubbard model: a reminiscience}
	
	\bibitem{tasaki} 
	H. Tasaki
	\textit{The Hubbard model - an introduction and selected rigorous results}
	
	\bibitem{jafari}
	Seyed A. Jafari 
	\textit{Introduction to Hubbard Model and Exact Diagonalization}
	
	\bibitem{gabrielsson}
	Anders F. Gabrielsson
	\textit{Quantum Monte Carlo Simulations of
		the Half-filled Hubbard Model}
	
	\bibitem{luu}
	Thomas Luu, November 2017
	\textit{Fermions and Computers}
	

\end{thebibliography}
\clearpage
\section{Appendix A - Analytical diagonalization of the 2-site model}
The matrix for the Hubbard Hamiltonian in the basis of the 16 states $\ket{i}, i\in {1,...,16}$ shown in section \ref{analytic} reads\footnote{note that due to particle conservation the matrix splits up into smaller blocks of same particle number}
\begin{equation*}
	H=
\begin{pmatrix}
	0&0&0&0&0&0&0&0&0&0&0&0&0&0&0&0\\
	0&-U/2&0&0&-t&0&0&0&0&0&0&0&0&0&0&0\\
	0&0&-U/2&0&0&-t&0&0&0&0&0&0&0&0&0&0\\
	0&0&0&0&0&0&0&0&-t&-t&0&0&0&0&0&0\\
	0&-t&0&0&-U/2&0&0&0&0&0&0&0&0&0&0&0\\
	0&0&-t&0&0&-U/2&0&0&0&0&0&0&0&0&0&0\\
	0&0&0&0&0&0&0&0&-t&-t&0&0&0&0&0&0\\
	0&0&0&0&0&0&0&0&-U&0&0&0&0&0&0&0\\
	0&0&0&-t&0&0&-t&0&-U&0&0&0&0&0&0&0\\
	0&0&0&-t&0&0&-t&0&0&-U&0&0&0&0&0&0\\
	0&0&0&0&0&0&0&0&0&0&-U&0&0&0&0&0\\
	0&0&0&0&0&0&0&0&0&0&0&-U/2&0&-t&0&0\\
	0&0&0&0&0&0&0&0&0&0&0&0&-U/2&0&-t&0\\
	0&0&0&0&0&0&0&0&0&0&0&-t&0&-U/2&0&0\\
	0&0&0&0&0&0&0&0&0&0&0&0&-t&0&-U/2&0\\
	0&0&0&0&0&0&0&0&0&0&0&0&0&0&0&0
\end{pmatrix}
\end{equation*}
Diagonalization gives 
\begin{align*}
	H_D= \text{diag}\biggl(0,0,0,-U,-U,-U, -\frac{U}{2}+t, -\frac{U}{2}+t,-\frac{U}{2}+t,-\frac{U}{2}+t,\\,-\frac{U}{2}-t,-\frac{U}{2}-t,-\frac{U}{2}-t,-\frac{U}{2}-t,-\frac{U}{2}-\sqrt{\frac{U^2}{4}+4t^2},-\frac{U}{2}+\sqrt{\frac{U^2}{4}+4t^2}\biggr)
\end{align*}
with the set of eigenstates $\ket{n_i}$:
\begin{align*}
	\ket{n_1}&= \ket{1}\hspace{2cm}\\
	\ket{n_2}&=-\frac{1}{\sqrt{2}} \ket{4} + \frac{1}{\sqrt{2}}\ket{7} \hspace{2cm}\\
	\ket{n_3}&=-\frac{1}{\sqrt{2}} \ket{14} + \frac{1}{\sqrt{2}}\ket{7}\\
	\ket{n_4}&=-\frac{1}{\sqrt{2}} \ket{12} + \frac{1}{\sqrt{2}}\ket{13}\\ \hspace{2cm}
	\ket{n_5}&=\ket{11} \\
	\ket{n_6}&=-\frac{1}{\sqrt{2}} \ket{9} + \frac{1}{\sqrt{2}}\ket{10}\\
	\ket{n_7}&= \ket{8} \\
	\ket{n_8}&=-\frac{1}{2} \ket{12} - \frac{1}{{2}}\ket{13}  + \frac{1}{{2}}\ket{14}  + \frac{1}{{2}}\ket{15}\\
	\ket{n_9}&=-\frac{1}{\sqrt{2}} \ket{3} + \frac{1}{\sqrt{2}}\ket{6}\\
	\ket{n_{10}}&=-\frac{1}{\sqrt{2}} \ket{2} + \frac{1}{\sqrt{2}}\ket{5}\\
	\ket{n_{11}}&=\frac{1}{\sqrt{2}} \ket{3} + \frac{1}{\sqrt{2}}\ket{6}\\
	\ket{n_{12}}&=\frac{1}{\sqrt{2}} \ket{2} + \frac{1}{\sqrt{2}}\ket{5}\\
	\ket{n_{13}}&=\frac{1}{2} \ket{12} + \frac{1}{{2}}\ket{13}  + \frac{1}{{2}}\ket{14}  + \frac{1}{{2}}\ket{15}\\
	\ket{n_{14}}&= -\frac{2t}{D_-} \ket{4} -\frac{2t}{D_-} \ket{7} +\frac{2t(\frac{U}{2}+\sqrt{\frac{U^2}{4}+4t^2})}{D_-} \ket{9} +\frac{2t(\frac{U}{2}+\sqrt{\frac{U^2}{4}+4t^2})}{D_-} \ket{10}\\
	\ket{n_{15}}&= +\frac{2t}{D_+} \ket{4} +\frac{2t}{D_+} \ket{7} +\frac{2t(-\frac{U}{2}+\sqrt{\frac{U^2}{4}+4t^2})}{D_+} \ket{9} +\frac{2t(-\frac{U}{2}+\sqrt{\frac{U^2}{4}+4t^2})}{D_+} \ket{10}\\
	\ket{n_{16}}&=\ket{16}
\end{align*}
where 
\begin{equation*}
	D_{\pm}= \biggl(\frac{U}{2}\pm\sqrt{\frac{U^2}{4}+4t^2}\biggr)\sqrt{\biggl|\frac{64t^2}{\frac{U^2}{4}+4t^2}\biggr|^2+2}
\end{equation*}
\clearpage
\noindent Finally the nonvanishing components of the annihilation operators $(a^{x}_\sigma)_{i,j}$ in matrix form are
\begin{equation*}
	(a^1_\downarrow)_{6,1}=(a^1_\downarrow)_{7,5}=(a^1_\downarrow)_{9,2}=(a^1_\downarrow)_{11,3}=(a^1_\downarrow)_{12,8}=(a^1_\downarrow)_{13,10}=(a^1_\downarrow)_{15,4}=(a^1_\downarrow)_{16,14}=1 
\end{equation*}
\begin{equation*}
	(a^1_\uparrow)_{5,1}=-(a^1_\uparrow)_{7,6}=(a^1_\uparrow)_{8,2}=(a^1_\uparrow)_{10,3}=(a^1_\uparrow)_{61}=-(a^1_\uparrow)_{12,9}=-(a^1_\uparrow)_{13,11}=-(a^1_\uparrow)_{16,15}=1
\end{equation*}
\begin{equation*}
	(a^2_\downarrow)_{3,1}=(a^2_\downarrow)_{4,2}=(a^2_\downarrow)_{10,5}=(a^2_\downarrow)_{11,6}=(a^2_\downarrow)_{13,7}=(a^2_\downarrow)_{14,8}=(a^2_\downarrow)_{15,9}=(a^2_\downarrow)_{16,12}=1
\end{equation*}
\begin{equation*}
	(a^2_\uparrow)_{6,1} =(a^2_\uparrow)_{2,1}=-(a^2_\uparrow)_{4,3}=(a^2_\uparrow)_{8,5}=(a^2_\uparrow)_{9,6}=(a^2_\uparrow)_{12,7}=-(a^2_\uparrow)_{14,10}=-(a^2_\uparrow)_{15,11}=-(a^2_\uparrow)_{16,13}=1
\end{equation*}

