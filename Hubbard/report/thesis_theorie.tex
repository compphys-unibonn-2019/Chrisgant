%==============================================================================
\chapter{Theoretische Einführung}
\label{sec:theo}
%==============================================================================

\section{Hubbard-Modell}
The Hubbard model, named after british Physicist John Hubbard, is a model mostly used in Condensed Matter Physics. It describes matter as a lattice on which sites particles e.g. electrons can hop to another lattice site, while in case of fermions they must, of course, respect the Pauli principle. Furthermore a Coulomb repulsion is assumed between electrons at the same lattice site. Figure \ref{fig:pic2} shows the lattice for a 2D material schematically. The onsite Coulomb repulsion is a feature that distinguishes the Hubbard model from most other models and enables it for instance to describe the transition from a metal to an insulator in certain materials. The Hamilton operator of the Hubbard model reads \cite{Hubbard}:
\begin{equation}\label{Hubbard_original}
H = \sum_{i,j}\sum_{\sigma}T_{ij} c_{i\sigma}^\dag c_{j\sigma} + \frac{I}{2} \sum_{i}n_{i\uparrow}n_{i\downarrow} - \frac{I}{2}Nn^2
\end{equation}
where $ c^\dag $,  $ c $ and $ n = c^\dag c $ denote the creation operator, annihilation operator and occupation number operator respectively. Hopping from one lattice to another therefore corresponds to annihilating a particle at the intial lattice site and creating one at the final site. The Parameter I represents the potential energy of the Coulomb repulsion. The matrix element $ T_{ij} $ represents the kinetic energy when hopping from lattice site $i$ to lattice site $j$ and would in general depends on the orbital overlap of both sites, but an aditional assumption is made for the particles to only be able to hop to neighboring lattice sites.  That means for a homogeneous lattice $ T_{ij} \approx -t\delta_{\langle ij \rangle} $\cite{schoett2014}. This assumption simplifies the matrix a lot. Also writing  $ U = \frac{I}{2} $ and omitting the constant yields the simpler form of the Hamiltonian \cite{Rickayzen,schoett2014}:
\begin{equation}\label{Hubbard_standard}
H = \underbrace{-t\sum_{\langle ij\rangle,\sigma}\left( c_{i\sigma}^\dag c_{j\sigma} + c_{j\sigma}^\dag c_{i\sigma}\right)}_{\substack{H_0}}  + \underbrace{U \sum_{i}n_{i\uparrow}n_{i\downarrow}}_{\substack{H_I}}
\end{equation}
A pictorial representation of the meaning of the parameters $ t $ and $ U $ is illustrated in \ref{fig:pic1} with $t$ being a tunneling energy and $U$ being the potential energy of two particles on the same site. 
$ H_I $ defines the Coulomb part of the Hamiltonian and $H_0$ the hopping part.
Der nicht interagierende Term $ H_0 $ lässt sich durch eine Fourier-Transformation in den Impulsraum diagonalisieren \cite{schoett2014}
\begin{equation}\label{Hubbard_momentum}
H = \sum_{\boldsymbol{k} \sigma} \epsilon_{\boldsymbol{k}} c_{\boldsymbol{k}\sigma}^\dag c_{\boldsymbol{k}\sigma}\\
\end{equation}
$ \epsilon_{\boldsymbol{k}} $ ist die Dispersion der Elektronen. Die Erzeugungs- und Vernichtungsoperatoren wirken nun nicht mehr auf zwei verschiedene Gitter Plätze sondern das Elektron mit dem Impuls $ \boldsymbol{k} $. Sie können einfach als Wannier-Funktion
\begin{equation}\label{Wannier}
\psi_{\boldsymbol{k}} \left( \boldsymbol{x}\right) = \frac{1}{\sqrt{N}}\sum_{j} e^{i\boldsymbol{k}\boldsymbol{R}_j} \phi(\boldsymbol{x}-\boldsymbol{R}_j)
\end{equation} 
der lokalen Operatoren berechnet werden
\begin{equation}\label{c_momentum}
c_{\boldsymbol{k}\sigma} = \frac{1}{\sqrt{N}}\sum_{j} e^{i\boldsymbol{k}\boldsymbol{R}_j} c_{j\sigma}, \qquad c_{\boldsymbol{k}\sigma}^\dagger = \frac{1}{\sqrt{N}}\sum_{j} e^{-i\boldsymbol{k}\boldsymbol{R}_j} c_{j\sigma}^\dagger
\end{equation}\cite{Hubbard}  
Durch Einsetzen dieser in \eqref{Hubbard_momentum} erhält man für $ T_{ij} $ aus \eqref{Hubbard_original}
\begin{equation}\label{T}
T_{jl} = \frac{1}{N} \sum_{\boldsymbol{k}}\epsilon_{\boldsymbol{k}} e^{i\boldsymbol{k}\left( \boldsymbol{R}_j - \boldsymbol{R}_l\right) }
\end{equation}

\begin{figure}[h!]
	\begin{subfigure}{.5 \textwidth}
		\centering
		\includegraphics[width=0.8\linewidth]{../figs/graphics/pic2}
		\caption{\cite{FigHubbardModel2}}
		\label{fig:pic2}
	\end{subfigure}
	\begin{subfigure}{.5 \textwidth}
		\centering
		\includegraphics[width=0.8\linewidth]{../figs/graphics/pic1}
		\caption{\cite{FigHubbardModel1}}
		\label{fig:pic1}
	\end{subfigure}
\caption[Veranschaulichung des Hubbard-Modells]{(a): Hier sieht man ein zweidimensionales quadratisches Gitter, die orangen Punkte symbolisieren Atome mit jeweils einem freien Orbital und die Gitterlinien zeigen die erlaubten Sprünge der Elektronen (hier als Pfeile dargestellt)\\(b): Diese Abbildung zeigt das Potential eines eindimensionalen Atom-Gitters, der Parameter t steht für die kinetische Energie eines Elektrons (rote und blaue Punkte) beim Sprung zu Nachbaratom, der Parameter U steht für die Potentielle Energie zweier Elektronen (up \& down) die sich gegenseitig abstoßen.}
\label{HubbardFigs}
\end{figure}


%nächster Absatz etwas unsicher, schau mal drüber

\subsection{Symmetries of the Hamiltonian}
From the above defintion \eqref{Hubbard_standard} it becomes clear that the total particle number $ \hat{N} $ as well as the total spin $ \hat{S}^2 $ and its z-component $ \hat{S}_z $ are conserved quantities. This means that in a suitable basis the matrix will decouple into smaller block matrices along the diagonal. This enables to look only on the block of most interest which in our case is the half-filled case. 

Wie man der Definition des Hamilton-Operators \eqref{Hubbard_standard} entnehmen kann bleiben die Teilchenzahl $ \hat{N} $, der gesamt Spin $ \hat{S}^2 $ und die Z-Komponente des Spins $ \hat{S}_z $ bei der Anwendung des Hamilton-Operators erhalten. Das führt dazu, dass die Matrix des Operators (bei geeigneter Wahl der Basis) aus Blockmatrizen für feste $ N $ entlang der Diagonalen besteht, welche wiederum den selben Aufbau mit Blöcken für feste $ S_z $ haben. Diese Eigenschaft ermöglicht es die Basis zu unterteilen, sodass man sich auf die wichtigsten Zustände beschränken kann. Es ist beispielsweise sehr unwahrscheinlich, dass alle Spins in dieselbe Richtung zeigen. Der interessanteste Block ist der mit halber Füllung und neutraler Spin-Ausrichtung, da dieser über die meisten Zustände verfügt und außerdem den Grundzustand enthält.


Eine weitere Eigenschaft, die man sich zu Nutze machen kann ist die Translations-Invarianz. Wenn man ein Gitter mit periodischen Grenzen betrachtet, also den Elektronen am Ende des Gitters erlaubt direkt zum gegenüberliegenden Ende zu springen, sollte klar sein, dass eine Translation aller Elektronen in eine Richtung nichts an den Physikalischen Eigenschaften des Zustandes ändert, da nur die Anordnung der Elektronen für diese verantwortlich ist, der Hamilton-Operator ist unabhängig von den Positionen einzelner Elektronen. Daraus folgt, dass einige der Basiszustände äquivalent sind und sich als Translation eines anderen Zustands schreiben lassen. Der Translationsoperator $ T $ verschiebt alle Elektronen um einen Gitterplatz. Nach einem kompletten Durchgang ist man wieder am Anfangszustand angekommen $ T^N\ket{a} = \ket{a} $. Da die verschiedenen Translationen äquivalent sind, gilt auch für die Eigenwerte von $ T $: $ \lambda^N = 1 $, daraus folgt $ \lambda = e^{ik} $ mit $ k = m\frac{2\pi}{N} $ für $ m \in \mathbb{N}, m < N $. Um diese Kriterien zu erfüllen wird der Impuls-Zustand $ \ket{a(k)} $ als 
\begin{equation}\label{momentumstate}
\ket{a(k)} = \frac{1}{\sqrt{N_a}}\sum_{r=0}^{N-1}e^{-ikr}T^r\ket{a}
\end{equation}
definiert, sodass $ T\ket{a(k)} = e^{ik}\ket{a(k)}$ gilt, wobei $ \ket{a} $ ein Referenzzustand aus der lokalen Basis ist und $ N_a $ für die Normierung sorgt \cite{Sandvik}. Der Startzustand wird in einigen Fällen schon vor $ N $ Verschiebungen wieder erreicht, dadurch werden die möglichen $ k $ eingeschränkt und die Normierung muss beachtet werden. Dazu wird $ R_a $ als kleinst mögliche Lösung von $ T^{R_a}\ket{a} = \ket{a} $ gewählt. Dann gilt
\begin{equation}\label{k}
k = m\frac{2\pi}{R_a},\qquad m \in \mathbb{N},\: m < R_a
\end{equation}
und
\begin{equation}\label{N_a}
N_a = \frac{N^2}{R_a}.
\end{equation}
Der Vorteil dieser Basis ist, dass der Hamilton-Operator in noch kleinere Blöcke mit konstantem $ k $ aufgeteilt werden kann, da dieses ebenfalls erhalten ist.

\subsection{Analytische Lösung für den Grundzustand}
Der Grundzustand des Hubbard-Modells, ist für diese Arbeit von großer Bedeutung, da im Photoemmissions- und -absorptionsspektrum die Energiedifferenz eines Zustandes zum Grundzustand auf der x-Achse aufgetragen wird und die Grundzustandsenergie somit in jedem Eintrag steckt. Da die direkte Lösung des Hubbard-Modells sogar im eindimensionalen Fall schon auf sehr kleine Gitter beschränkt ist wird die Grundzustandsenergie nicht mit den realen Eigenschaften des Materials übereinstimmen, jedoch kann man eine Konvergenz mit steigender Gitterlänge gegen diese analytische Lösung erwarten
\begin{equation}\label{analytisch}
E_0(U)=-4L\int_{0}^{\infty}\frac{J_0(\omega)J_1(\omega)}{\omega\left( 1+\exp\left( \omega\frac{U}{2}\right) \right) }d\omega
\end{equation}\cite{Monien}
Hier steht $ L $ für die Anzahl der Gitterplätze und $ J_n(\omega) $ für die Besselfunktion erster Gattung.




\section{Spektralfunktion}
Spektralfunktionen werden in der Physik genutzt um die physikalischen Eigenschaften (z.B. die Zustandsdichte) eines Materials darzustellen.\cite{Lin} Das Ergebnis hat die Form eines Spektrums, wie man es auch experimentell aufnehmen kann, daher eignet sich die Spektralfunktion gut um eine Theorie mit tatsächlichen Messwerten zu vergleichen.
Sobald man die Greensche Funktion kennt, kann ihre Spektralfunktion wie folgt bestimmt werden
\begin{equation}\label{G2A}
A(\omega)= \lim\limits_{\eta \rightarrow 0}\,-\frac{1}{\pi}\Im\left[ G(\omega+i\eta)\right] 
\end{equation}
Um aus einer Spektralfunktion ihre Greensche Funktion zu extrahieren muss man eine Hilberttransformation anwenden
\begin{equation}\label{A2G}
G(z) = \int_{-\infty}^{\infty}\frac{A(\zeta)}{z- \zeta}\dd{\zeta}
\end{equation}
\cite{Rickayzen,schoett2014}
Eine Spektralfunktion kann als Summe von Deltafunktionen geschrieben werden, deren Gewichtung vom Matrixelement zwischen dem gestörten Grundzustand $ B\ket{\psi_0} $ und dem angeregten Zustand $ \ket{\psi_m} $ gegeben ist.
\begin{equation}\label{deltasum}
A(\omega) =\sum_{m}\abs{\bra{\psi_m}B\ket{\psi_0}}^2\delta\left( \omega-(\lambda_m-\lambda_0)\right) 
\end{equation}
Die kombinierte Spektralfunktion aus Photoemission und Photoabsorption, welche der Zustandsdichte entspricht wenn beide Operatoren auf das selbe Elektron wirken, verwendet als Störoperator $ B $ den Erzeugungs- und Vernichtungsoperator. Man erhält für diese also folgende Formel
\begin{equation}\label{rho}
\rho(\omega) = \sum_{m}\abs{\bra{\psi_m}c^\dagger_i\ket{\psi_0}}^2\delta\left( \omega -\lambda_m +\lambda_0\right) + \sum_{n}\abs{\bra{\psi_n}c_i\ket{\psi_0}}^2\delta\left( \omega +\lambda_n + \lambda_0\right)
\end{equation}


%denke ab hier brauchen wir nicht mehr?

\section{Greensche Funktion}
Die Greensche Funktion kann allgemein zur Lösung von Differentialgleichungen genutzt werden, indem sie für einen bestimmten linearen Differentialoperator $ L $ die Deltafunktion ergibt
\begin{equation}\label{G1}
LG(x,y) = \delta(x-y)
\end{equation}
Dies ermöglicht es, sozusagen das Inverse des Differentialoperators zu berechnen. Wenn $ L f(x) = g(x) $ gilt und sowohl $ L $ als auch $ g(x) $ bekannt sind, ist es dennoch nicht trivial $ f(x) $ zu bestimmen, da man den Operator nicht einfach invertieren kann. Mit der Greenschen Funktion kann dies durch folgenden \emph{Trick} dennoch gelingen
\begin{align*}
L f(x) &= g(x)\\
L f(x) &= \int \delta(x-y)g(y)\dd{y}\\
L f(x) &= \int L G(x,y)g(y)\dd{y}\\
\end{align*}
\begin{equation}\label{Hilbert}
\Rightarrow f(x)  = \int G(x,y) g(y) \dd{y}
\end{equation}
In der Physik der kondensierten Materie haben Greensche Funktionen häufig nur noch wenig Ähnlichkeiten mit ihrer ursprünglichen mathematischen Definition.
In \eqref{G2A} wird die Spektralfunktion aus der Greenschen Funktion erstellt. Wie man in \eqref{deltasum} sieht lässt sich die Spektralfunktion als Summe aus Deltafunktionen darstellen, beim Vergleich mit \eqref{G1} stellt man also fest, dass $ \lim\limits_{\eta \rightarrow 0}\,-\frac{1}{\pi}\Im $ als Operator $ L $ infrage kommt (zusätzlich muss natürlich $ i\eta $ zu $ \omega $ hinzu addiert werden). Unter den verschiedenen Darstellungen der Deltafunktion befindet sich auch 
\begin{equation}\label{delta}
\delta(x - y) =\lim\limits_{\eta \rightarrow 0}\,-\frac{1}{\pi}\Im\left[ \frac{1}{x - y +i\eta}\right] 
\end{equation}
wodurch klar ist, dass
\begin{equation}\label{G}
G(x,y) = \frac{1}{x - y}
\end{equation}
gilt. Diese kann einfach mit \eqref{rho} in \eqref{Hilbert} eingesetzt werden und man erhält die Greensche Funktion für die Zustandsdichte
\begin{equation}\label{Grho}
G(\omega) = G_{ii}(\omega) = \frac{\sum_{m}\abs{\bra{\psi_m}c^\dagger_i\ket{\psi_0}}^2}{\omega +\lambda_0 -\lambda_m } + \frac{\sum_{n}\abs{\bra{\psi_n}c_i\ket{\psi_0}}^2}{\omega + \lambda_0 +\lambda_n}
\end{equation}
%\begin{equation}\label{TAB}
%G(\boldsymbol{r}_1,\tau_1,\boldsymbol{r}_2,\tau_2) = \left\langle T\,A(\boldsymbol{r}_1,\tau_1)B(\boldsymbol{r}_2,\tau_2)\right\rangle 
%\end{equation}
%Hier ist $ T $ der Zeitordnungsoperator und $ A $ und $ B $ sind beliebige Quantenmechanische Operatoren. Der Einfachheit halber wird in dieser Arbeit von hier an die Greensche Funktion für das Photoemmissions- und -absorptionsspektrum bei $ T = \SI{0}{\kelvin} $ behandelt, da man in der Regel an dieser interessiert ist. Für diese Greensche Funktion wählt man $ c_i(\tau) $ und $ c_j^\dagger(0) $ als Operatoren.
%\begin{align*}
%G_{ij}(\tau) &= \left\langle T\,c_i(\tau)c_j^\dagger(0)\right\rangle \\
% &= \Theta(\tau)\bra{0}c_i(\tau)c_j^\dagger(0)\ket{0} - \Theta(-\tau)\bra{0}c_j^\dagger(0)c_i(\tau)\ket{0}\\
% &= \Theta(\tau)\bra{0}e^{iH\tau}c_ie^{-iH\tau}c_j^\dagger\ket{0} - \Theta(-\tau)\bra{0}c_j^\dagger e^{iH\tau}c_ie^{-iH\tau}\ket{0}\\
% &= \Theta(\tau)\bra{0}c_ie^{-i(H-E_0)\tau}c_j^\dagger\ket{0} - \Theta(-\tau)\bra{0}c_j^\dagger e^{i(H-E_0)\tau}c_i\ket{0}\\
% &= \Theta(\tau)\sum_{m}e^{-i(E_m-E_0)\tau}\bra{0}c_i\ket{m}\bra{m}c_j^\dagger\ket{0} - \Theta(-\tau)\sum_{m}e^{i(E_m-E_0)\tau}\bra{0}c_j^\dagger\ket{m}\bra{m} c_i\ket{0}\\
% &= \sum_{m}e^{-i(E_m-E_0)\abs{\tau}}\left[ \Theta(\tau)\bra{0}c_i\ket{m}\bra{m}c_j^\dagger\ket{0} - \Theta(-\tau)\bra{0}c_j^\dagger\ket{m}\bra{m} c_i\ket{0}\right] \\
%\end{align*}
%Diese Funktion kann nun Fourier-transformiert werden
%\begin{align*}
%	\hat{G}(\omega) &= \int_{-\infty}^{\infty}
%\end{align*}
%\cite{Rickayzen,schoett2014,FZJ2017}
diese lässt sich ebenfalls schreiben als
\begin{equation}\label{GreensFunctionDOS}
G(\omega) = \bra{0}c\frac{1}{(\omega + \lambda_0) I - H}c^\dagger\ket{0} + \bra{0}c^\dagger\frac{1}{(\omega + \lambda_0) I + H}c\ket{0}
\end{equation}\cite{Lu,Stover,Ulm}

\section{Anderson-Impurity-Model}
\begin{equation}\label{key}
H_{A} = \epsilon_d \sum_{\sigma} d_{\sigma}^\dag d_{\sigma} +\sum_{\sigma,l}\epsilon_l c_{l\sigma}^\dag c_{l\sigma} + Un_{d\uparrow}n_{d\downarrow} + \sum_{\sigma,l}\left( V_l c_{l\sigma}^\dag d_{\sigma} + V_l^* d_{\sigma}^\dag c_{l\sigma}\right) 
\end{equation}



\section{DMFT}


\section{Bethe-Gitter}

\section{Lanczos Methode}
Um die Eigenwerte der immer noch zu großen Matrix zu berechnen, kann man die Lanczos Methode verwenden. Diese erzeugt einen invarianten Unterraum der großen Matrix A, bei diesem Unterraum handelt es sich um einen Krylov Unterraum ($b,Ab,A^2b,...,A^{m-1}b$). Der letzte Vektor liegt dabei nicht mehr im Unterraum, aber für ausreichend große m  ist dieser ungefähr proportional zum Vorletzten, da dieser gegen einen Eigenvektor strebt, daher liegt der letzte Vektor näherungsweise auch im Unterraum. Die Lanczos Methode gibt eine tridiagonale Matrix T für diesen Unterraum aus, deren Eigenwerte auch Eigenwerte von A sind.\cite{Lin, FZJ}
\begin{equation}
AQ = QT\;,\;Ty=\lambda y \Rightarrow A(Qy)=\lambda(Qy)
\end{equation} 
In der Regel sind es die größten und kleinsten Eigenwerte von A, die man in Matrix T findet, daher eignet sich die Lanczos Methode um die Energie des Grundzustands, also den kleinsten Eigenwert zu finden. Die Matrix T wird iterativ mit folgender Rekursions-Beziehung gefunden, 
\begin{equation}
Aq_j = \beta_{j-1}q_{j-1}+\alpha_jq_j+\beta_{j}q_{j+1}
\end{equation}
wobei der Vektor $\alpha$ der Hauptdiagonalen und der Vektor $\beta$ den Nebendiagonalen der Matrix T entspricht.\cite{Lloyd}







%%% Local Variables: 
%%% mode: latex
%%% TeX-master: "../mythesis"
%%% End: 
